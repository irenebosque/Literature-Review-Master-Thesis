%%%%%%%%%%%%%%%%%%%%%%%%%%%%%%%%%%%%%%%%%
% University Assignment Title Page 
% LaTeX Template
% Version 1.0 (27/12/12)
%
% This template has been downloaded from:
% http://www.LaTeXTemplates.com
%
% Original author:
% WikiBooks (http://en.wikibooks.org/wiki/LaTeX/Title_Creation)
%
% License:
% CC BY-NC-SA 3.0 (http://creativecommons.org/licenses/by-nc-sa/3.0/)
% 
% Instructions for using this template:
% This title page is capable of being compiled as is. This is not useful for 
% including it in another document. To do this, you have two options: 
%
% 1) Copy/paste everything between \begin{document} and \end{document} 
% starting at \begin{titlepage} and paste this into another LaTeX file where you 
% want your title page.
% OR
% 2) Remove everything outside the \begin{titlepage} and \end{titlepage} and 
% move this file to the same directory as the LaTeX file you wish to add it to. 
% Then add \input{./title_page_1.tex} to your LaTeX file where you want your
% title page.
%
%%%%%%%%%%%%%%%%%%%%%%%%%%%%%%%%%%%%%%%%%
%\title{Title page with logo}
%----------------------------------------------------------------------------------------
%	PACKAGES AND OTHER DOCUMENT CONFIGURATIONS
%----------------------------------------------------------------------------------------

\documentclass[11pt]{article}
\usepackage[english]{babel}
\usepackage[utf8x]{inputenc}
\usepackage{amsmath}
\usepackage{graphicx}
\setlength{\marginparwidth}{2cm}
\usepackage[colorinlistoftodos]{todonotes}
\usepackage{hyperref}
\usepackage{xcolor}
\usepackage[margin=1in]{geometry}
\usepackage{tikz}
\usetikzlibrary{matrix,chains,positioning,decorations.pathreplacing,arrows}
\usetikzlibrary{trees}
\usepackage{float}
\usepackage[gen]{eurosym}
\usepackage{tasks}
\usepackage{verbatim}
\usepackage[normalem]{ulem}
\useunder{\uline}{\ul}{}
\usepackage{dirtytalk} % for quotation marks :P
\usepackage{subcaption}
\usepackage{array,ragged2e}
\usepackage{comment}
\usepackage[hybrid]{markdown}
\usepackage{algorithm} 
\usepackage{algpseudocode} 
\usepackage{tabularx}
\newcommand{\quotes}[1]{``#1''}
\usepackage{breakcites}
\usepackage{listings}
\usepackage{soul}

%%%%%CODE SNIPPETS%%%
%New colors defined below
\definecolor{codegreen}{rgb}{0,0.6,0}
\definecolor{codegray}{rgb}{0.5,0.5,0.5}
\definecolor{codepurple}{rgb}{0.58,0,0.82}
\definecolor{backcolour}{rgb}{0.95,0.95,0.92}

%Code listing style named "mystyle"
\lstdefinestyle{mystyle}{
  backgroundcolor=\color{backcolour},   commentstyle=\color{codegreen},
  keywordstyle=\color{magenta},
  numberstyle=\tiny\color{codegray},
  stringstyle=\color{codepurple},
  basicstyle=\ttfamily\footnotesize,
  breakatwhitespace=false,         
  breaklines=true,                 
  captionpos=b,                    
  keepspaces=true,                 
  numbers=left,                    
  numbersep=5pt,                  
  showspaces=false,                
  showstringspaces=false,
  showtabs=false,                  
  tabsize=2
}

%"mystyle" code listing set
\lstset{style=mystyle}
%%%%%%%%%%%%%%%%%%%%



\begin{document}

\begin{titlepage}

\newcommand{\HRule}{\rule{\linewidth}{0.5mm}} % Defines a new command for the horizontal lines, change thickness here

\center % Center everything on the page
 
%----------------------------------------------------------------------------------------
%	HEADING SECTIONS
%----------------------------------------------------------------------------------------

\textsc{\LARGE TU Delft}\\[1.2cm] % Name of your university/college
\textsc{\Large Literature Review}\\[0.5cm] % Major heading such as course name

%----------------------------------------------------------------------------------------
%	TITLE SECTION
%----------------------------------------------------------------------------------------

\HRule \\[0.4cm]
{ \huge \bfseries  Towards Off-Policy Corrective Imitation Learning}\\[0.4cm] % Title of your document
\HRule \\[1.0cm]
 

 % Towards Off-Policy Corrective Imitation Learning
%----------------------------------------------------------------------------------------
%	AUTHOR SECTION
%----------------------------------------------------------------------------------------

\begin{minipage}{0.5\textwidth}
\begin{flushleft} \large
\emph{Author:}\\
Irene \textsc{Bosque} (5051487)\\
\end{flushleft}
\end{minipage}
~
\begin{minipage}{0.4\textwidth}
\begin{flushright} \large
\emph{Supervisors:} \\
Jens \textsc{Kober}\\ %
Rodrigo \textsc{Pérez-Dattari}\\ % Supervisor's Name
Carlos \textsc{Celemin} % Supervisor's Name
\end{flushright}
\end{minipage}\\[2cm]

% If you don't want a supervisor, uncomment the two lines below and remove the section above
%\Large \emph{Author:}\\
%John \textsc{Smith}\\[3cm] % Your name

%----------------------------------------------------------------------------------------
%	DATE SECTION
%----------------------------------------------------------------------------------------

{\large February 22 2021}\\[2cm] % Date, change the \today to a set date if you want to be precise

%----------------------------------------------------------------------------------------
%	LOGO SECTION
%----------------------------------------------------------------------------------------

\includegraphics[width=6cm]{Figures/TUDelft_logo.png} % Include a department/university logo - this will require the graphicx package
 
%----------------------------------------------------------------------------------------

\vfill % Fill the rest of the page with whitespace

\end{titlepage}

\tableofcontents

\newpage
%\listoffigures
%\listoftables
%\newpage

\input{A_abstract.tex}
\input{B_Introduction.tex}
\input{C_Content}
\section{Conclusions}


Sections \ref{section:Reinforcement-Learning}, \ref{section:Imitation-Learning} and \ref{section:Towards-off-policy-CIL} present the theoretical relevant  background for this master thesis. Specifically, section \ref{section:Reinforcement-Learning} introduces reinforcement learning and how it makes use of MDPs as a framework for sequential decision making problems. The review continues explaining what the terms on-policy and off-policy mean in RL with the help of SARSA and Q-learning algorithms. This section ends with the concept of experience replay, a technique  that is a critical part of this thesis. In section \ref{section:Imitation-Learning} we introduce the concept of imitation learning, a machine learning method that leverages human knowledge in the learning process being more efficient than pure autonomous learning approaches in complex real-world problems. We present the definitions found in the literature of the terms on-policy and off-policy in imitation learning together with our interpretation with which several IL algorithms are classified.
This literature review ends with section \ref{section:Towards-off-policy-CIL} where we explain in more detail the algorithm COACH and its deep version D-COACH. There, it is explained why D-COACH needs to be transformed into an off-policy algorithm to fully leverage experience replay. Finally, we present the proposal of how to carry on this transformation by introducing a new model in the D-COACH framework that will predict the human's feedback.




\newpage
\input{z_Appendix.tex}
\newpage
\bibliographystyle{apalike}
\bibliography{G_bibliography.bib}
\newpage
%\input{H_Appendix.tex}

%Comments can be added to the margins of the document using the \todo{Here's a comment in the margin!} todo command, as shown in the example on the right. You can also add inline comments too:

%\todo[inline, color=green!40]{This is an inline comment.}

% \input{E_Analysis.tex}
% \input{F_Discussion.tex}
\end{document}
